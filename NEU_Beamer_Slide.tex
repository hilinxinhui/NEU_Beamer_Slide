\documentclass[10pt,aspectratio=43,mathserif]{beamer}		
%设置为 Beamer 文档类型,设置字体为 10pt,长宽比为16:9,数学字体为 serif 风格

%%%%-----导入宏包-----%%%%
\usepackage{neu}
\usepackage{xeCJK}
\usepackage{amsmath,amsfonts,amssymb,bm}
\usepackage{color}
\usepackage{graphicx,hyperref,url}
\usepackage{latexsym,amsmath,xcolor,multicol,booktabs,calligra}
\usepackage{graphicx,pstricks,listings,stackengine}
%%%%%%%%%%%%%%%%%%

\definecolor{neu_blue}{RGB}{42,69,140}

%%%%-----设置字体-----%%%%
%Windows和Mac OS下都可用
\setsansfont[Path=fonts/]{Helvetica}

%\setsansfont{Times New Roman}

%仅Windows可用
%\setCJKmainfont{Hiragino Sans GB W3}

%仅Mac OS下可用
%\setCJKmainfont{Songti SC}

%设置 Beamer 主题
\beamertemplateballitem

% \AtBeginSection[]
% {
%   \begin{frame}<beamer>
%     \frametitle{\textbf{目录}}
%     \textbf{\tableofcontents[currentsection]}
%   \end{frame}
% }

% defs
\def\cmd#1{\texttt{\color{red}\footnotesize $\backslash$#1}}
\def\env#1{\texttt{\color{blue}\footnotesize #1}}
\definecolor{deepblue}{rgb}{0,0,0.5}
\definecolor{deepred}{rgb}{0.6,0,0}
\definecolor{deepgreen}{rgb}{0,0.5,0}
\definecolor{halfgray}{gray}{0.55}

\lstset{
    basicstyle=\ttfamily\small,
    keywordstyle=\bfseries\color{deepblue},
    emphstyle=\ttfamily\color{deepred},    % Custom highlighting style
    stringstyle=\color{deepgreen},
    numbers=left,
    numberstyle=\small\color{halfgray},
    rulesepcolor=\color{red!20!green!20!blue!20},
    frame=shadowbox,
}

%%%%----首页信息设置----%%%%
\title[NEU Beamer Theme]{\fontsize{13pt}{18pt}\selectfont {NEU Beamer Theme}}
\subtitle{\fontsize{9pt}{14pt}\selectfont \textbf{利用Beamer制作幻灯片}}			
%%%%----标题设置

\author[林新辉]{
  林新辉 \\\medskip
  {\small {hilinxinhui@gmail.com}}}
  % \author[R. Song]{
  % Bradley Reaves, Nolen Scaife, Dave Tian, Logan Blue, \\
  % Patrick Traynor and Kevin R.B. Butler \\\medskip
  % {\small {\{reaves, scaife, daveti, bluel\}@ufl.edu}} \\
  % {\small {\{traynor, butler\}@cise.ufl.edu}}}
%%%%----个人信息设置

\institute[<lab>]{
  School of Computer Science and Engineering \\
  Northeastern University, China}
%%%%----机构信息

\date[\today]{
 \today}
%%%%----日期

\begin{document}

% 标题页
\begin{frame}
\titlepage
\end{frame}

% 目录页
\section*{目录}
\begin{frame}
    \frametitle{\textbf{目录}}
		\textbf{\tableofcontents}
\end{frame}

\section{研究背景}
\begin{frame}{用Beamer很高大上?}
  % \begin{itemize}[<+-| alert@+>] % 当然,除了alert,手动在里面插 \pause 也行
    \begin{itemize}
      \item 大家都会\LaTeX{},NEU也有了自己的Beamer主题!
      \item 中文内容请使用 Xe\LaTeX{} 编译
      \item 即将上传Overleaf,敬请期待
      % \item Overleaf项目地址位于 \url{},可以直接使用
      \item 本项目已开源,GitHub仓库见 \\ \url{https://github.com/hilinxinhui/NEU_Beamer_Slide} \\ 欢迎Star、Fork,欢迎提交Issues和PRs
  \end{itemize}
\end{frame}

\section{研究现状}

\subsection*{Beamer}

\begin{frame}{Beamer}
  \begin{itemize}
    \item 什么是Beamer? \\ \url{https://en.wikipedia.org/wiki/Beamer_(LaTeX)}
    \item 如何使用Beamer? \\ \url{https://www.overleaf.com/learn/latex/Beamer}
    \item 在这里可以找到若干Beamer模板 \\ \url{https://www.overleaf.com/gallery}
  \end{itemize}
\end{frame}

\begin{frame}{Why \LaTeX{} ?}
  \begin{itemize}
    \item \LaTeX 广泛用于学术界,期刊会议论文模板
\end{itemize}
\begin{table}[h]
    \centering
    \begin{tabular}{c|c}
        Microsoft\textsuperscript{\textregistered}  Word & \LaTeX \\
        \hline
        文字处理工具 & 专业排版软件 \\
        容易上手,简单直观 & 容易上手 \\
        所见即所得 & 所见即所想,所想即所得 \\
        高级功能不易掌握 & 进阶难,但一般用不到 \\
        处理长文档需要丰富经验 & 和短文档处理基本无异 \\
        花费大量时间调格式 & 无需担心格式,专注内容 \\
        公式排版差强人意 & 尤其擅长公式排版 \\
        二进制格式,兼容性差 & 文本文件,易读、稳定 \\
        付费商业许可 & 自由免费使用 \\
    \end{tabular}
\end{table}
\end{frame}

\subsection*{Related Efforts}

\begin{frame}{Related Efforts}
  \begin{itemize}
    \item 样式依据SEU-Beamer-Slide改编,感谢原作者的设计 \\ \url{https://github.com/TouchFishPioneer/SEU-Beamer-Slide}
    \item NEU-Beamer-Slide提供了校徽矢量图等各种资源文件 \\ \url{https://github.com/zhouyanasd/NEU-Beamer-Slide}
    \item 内容编排参考THU-Beamer-Theme \\ \url{https://github.com/tuna/THU-Beamer-Theme}
  \end{itemize}
\end{frame}

\section{研究内容}

\begin{frame}[fragile]{\LaTeX{} 常用命令}
  % \begin{exampleblock}{命令}
  %     \centering
  %     \footnotesize
  %     \begin{tabular}{llll}
  %         \cmd{chapter} & \cmd{section} & \cmd{subsection} & \cmd{paragraph} \\
  %         章 & 节 & 小节 & 带题头段落 \\\hline
  %         \cmd{centering} & \cmd{emph} & \cmd{verb} & \cmd{url} \\
  %         居中对齐 & 强调 & 原样输出 & 超链接 \\\hline
  %         \cmd{footnote} & \cmd{item} & \cmd{caption} & \cmd{includegraphics} \\
  %         脚注 & 列表条目 & 标题 & 插入图片 \\\hline
  %         \cmd{label} & \cmd{cite} & \cmd{ref} \\
  %         标号 & 引用参考文献 & 引用图表公式等\\\hline
  %     \end{tabular}
  % \end{exampleblock}
  % \begin{exampleblock}{环境}
  %     \centering
  %     \footnotesize
  %     \begin{tabular}{lll}
  %         \env{table} & \env{figure} & \env{equation}\\
  %         表格 & 图片 & 公式 \\\hline
  %         \env{itemize} & \env{enumerate} & \env{description}\\
  %         无编号列表 & 编号列表 & 描述 \\\hline
  %     \end{tabular}
  % \end{exampleblock}
\end{frame}

\begin{frame}[fragile]{\LaTeX{} 命令举例}
%   \begin{minipage}{0.5\linewidth}
% \begin{lstlisting}[language=TeX]
% \begin{itemize}
% \item A \item B
% \item C
% \begin{itemize}
%   \item C-1
% \end{itemize}
% \end{itemize}
% \end{lstlisting}
%   \end{minipage}\hspace{1cm}
%   \begin{minipage}{0.3\linewidth}
%       \begin{itemize}
%           \item A
%           \item B
%           \item C
%           \begin{itemize}
%               \item C-1
%           \end{itemize}
%       \end{itemize}
%   \end{minipage}
%   \medskip
%   \begin{minipage}{0.5\linewidth}
% \begin{lstlisting}[language=TeX]
% \begin{enumerate}
% \item 巨佬 \item 大佬
% \item 萌新
% \begin{itemize}
%   \item 瑟瑟发抖
% \end{itemize}
% \end{enumerate}
% \end{lstlisting}
%   \end{minipage}\hspace{1cm}
%   \begin{minipage}{0.3\linewidth}
%       \begin{enumerate}
%           \item 巨佬
%           \item 大佬
%           \item 萌新
%           \begin{itemize}
%               \item 瑟瑟发抖
%           \end{itemize}
%       \end{enumerate}
%   \end{minipage}
\end{frame}

\begin{frame}[fragile]{\LaTeX{} 数学公式}
%   \begin{columns}
%       \begin{column}{.55\textwidth}
% \begin{lstlisting}[language=TeX]
% $V = \frac{4}{3}\pi r^3$

% \[
% V = \frac{4}{3}\pi r^3
% \]

% \begin{equation}
% \label{eq:vsphere}
% V = \frac{4}{3}\pi r^3
% \end{equation}
% \end{lstlisting}
%       \end{column}
%       \begin{column}{.4\textwidth}
%           $V = \frac{4}{3}\pi r^3$
%           \[
%               V = \frac{4}{3}\pi r^3
%           \]
%           \begin{equation}
%               \label{eq:vsphere}
%               V = \frac{4}{3}\pi r^3
%           \end{equation}
%       \end{column}
%   \end{columns}
%   \begin{itemize}
%       \item 更多内容请看 \href{https://zh.wikipedia.org/wiki/Help:数学公式}{\color{purple}{这里}}
%   \end{itemize}
\end{frame}

\begin{frame}{\LaTeX{} 数学公式举例}
  % \begin{exampleblock}{无编号公式} % 加 * 
  %     \begin{equation*}
  %         J(\theta) = \mathbb{E}_{\pi_\theta}[G_t] = \sum_{s\in\mathcal{S}} d^\pi (s)V^\pi(s)=\sum_{s\in\mathcal{S}} d^\pi(s)\sum_{a\in\mathcal{A}}\pi_\theta(a|s)Q^\pi(s,a)
  %     \end{equation*}
  % \end{exampleblock}
  % \begin{exampleblock}{多行多列公式\footnote{如果公式中有文字出现,请用 $\backslash$mathrm\{\} 或者 $\backslash$text\{\} 包含,不然就会变成 $clip$,在公式里看起来比 $\mathrm{clip}$ 丑非常多。}}
  %     % 使用 & 分隔
  %     \begin{align}
  %         Q_\mathrm{target}&=r+\gamma Q^\pi(s^\prime, \pi_\theta(s^\prime)+\epsilon)\\
  %         \epsilon&\sim\mathrm{clip}(\mathcal{N}(0, \sigma), -c, c)\nonumber
  %     \end{align}
  % \end{exampleblock}
\end{frame}

% \begin{frame}{续:数学公式举例}
%   \begin{exampleblock}{编号多行公式}
%       % Taken from Mathmode.tex
%       \begin{multline}
%           A=\lim_{n\rightarrow\infty}\Delta x\left(a^{2}+\left(a^{2}+2a\Delta x+\left(\Delta x\right)^{2}\right)\right.\label{eq:reset}\\
%           +\left(a^{2}+2\cdot2a\Delta x+2^{2}\left(\Delta x\right)^{2}\right)\\
%           +\left(a^{2}+2\cdot3a\Delta x+3^{2}\left(\Delta x\right)^{2}\right)\\
%           +\ldots\\
%           \left.+\left(a^{2}+2\cdot(n-1)a\Delta x+(n-1)^{2}\left(\Delta x\right)^{2}\right)\right)\\
%           =\frac{1}{3}\left(b^{3}-a^{3}\right)
%       \end{multline}
%   \end{exampleblock}
% \end{frame}

\begin{frame}{插入图片和表格}
  % \begin{itemize}
  %     \item 矢量图 eps, ps, pdf
  %     \begin{itemize}
  %         \item METAPOST, pstricks, pgf $\ldots$
  %         \item Xfig, Dia, Visio, Inkscape $\ldots$
  %         \item Matlab / Excel 等保存为 pdf
  %     \end{itemize}
  %     \item 标量图 png, jpg, tiff $\ldots$
  %     \begin{itemize}
  %         \item 提高清晰度,避免发虚
  %         \item 应尽量避免使用
  %     \end{itemize}
  % \end{itemize}
  % \begin{figure}[htpb]
  %     \centering
  %     \includegraphics[width=0.2\linewidth]{assets/neu_logo.png}
  %     \caption{这个校徽就是矢量图}
  % \end{figure}
\end{frame}

\begin{frame}[fragile]{插入参考文献}
\end{frame}

\section{计划进度}
\begin{frame}
  \begin{itemize}
    \item 一月:完成文献调研
    \item 二月:复现并评测各种Beamer主题美观程度
    \item 三、四月:美化THU Beamer主题
    \item 五月:论文撰写
\end{itemize}
\end{frame}

\section*{参考文献}
\begin{frame}
\end{frame}

\section*{}
\begin{frame}
  \begin{center}
      {\Huge\calligra Thanks!}
  \end{center}
\end{frame}

\end{document}